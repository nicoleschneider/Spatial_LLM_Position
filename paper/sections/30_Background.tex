\section{Spatial Reasoning}
\label{section:background}

In this section we define the problem of spatial reasoning and describe how it presents a worthwhile challenge to address using LLMs.

Spatial reasoning is the task of reasoning about entities or objects that are located in space and possess spatial structure, or have spatial relationships with each other~\cite{Varzi2007}.
The abstract theories that define the core principles of spatial reasoning are commonly divided into classes including 
\textbf{(\textit{i})} \textit{mereology}, the theory of parthood relations;
\textbf{(\textit{ii})} \textit{topology}, the theory of qualitative spatial relations such as continuity and contiguity, and 
\textbf{(\textit{iii})} the \textit{theory of location proper}, which deals with the relationship between an entity and the spatial region it
occupies~\cite{Varzi2007}.
%
In this paper we discuss spatial reasoning tasks, which involve answering questions about certain relationships (termed spatial relations) between some set of objects in space.



\subsection{Spatial Entities.}
Spatial entities are the basic elements of spatial data.
They can be classed into three main types: \textbf{(\textit{i})} \textit{Points}, which consist of an $(x,y)$ coordinate pair in Cartesian space, \textbf{(\textit{ii})} \textit{Lines}, which represent the shortest path between two points, and \textbf{(\textit{iii})} \textit{Regions}, which represent the area inside a polyline joining several points. 
Point entities are typically used to represent locations in the world, like landmarks, towns, buildings, and objects.
Line entities are typically used to represent ways, like roadways, waterways, and railways.
Region entities are typically used to represent the extent of lakes, stadiums, large buildings, states, and counties.
Spatial entities represent a conceptualization of the the building blocks of the physical world, and so they often appear in questions that a person might be interested in asking an LLM, such as \textit{what} place has $x$ property or \textit{where} event $y$ occurred.


\subsection{Spatial Relations.}
\textit{Points, Lines} and \textit{Regions} relate to each other spatially through the following types of relationships~\cite{Carniel2020,Bertella2022,Carniel2023}: 
\textbf{(\textit{i})} \textit{Metric relations} that describe the distances between spatial entities (like ten miles, near, far), 
\textbf{(\textit{ii})} \textit{Topological relations} that describe how regions, lines, and points interact (like intersect, contain, touch), 
\textbf{(\textit{iii})} \textit{Directional relations} that describe how entities are positioned relatively in space (like north, left, behind), and
\textbf{(\textit{iv})} \textit{Order} relations that describe the cyclic order in which objects appear with respect to a central coordinate~\cite{Schwering2014}.
%
%Hybrid relations that combine two or more of the primitive spatial relations are also possible~\cite{Carniel2023}.
Spatial relations enable qualitative or quantitative descriptions of how physical places or objects in the world interact spatially.
To answer spatial questions accurately, an LLM must not only recognize spatial entities, but also interpret their spatial relations to infer information about the world, such as city $A$ is North of city $B$ and city $B$ is North of city $C$, therefore city $A$ is North of city $C$.






% \subsection{LLM Spatial Reasoning}
% How have people tried to use LLMs to aid/do it?

% What are the key challenges that haven't been addressed?
