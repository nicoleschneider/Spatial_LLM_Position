\section{Discussion}
\label{section:discussion}

\subsection{Toponym Resolution}



\subsection{Metric Relations}

\nrscomment{discuss w.r.t. previous work on near/far}

\citeauthor{Bhandari2023} prompt the LLMs to generate names of cities that are ``near'' or ``far'' from a provided reference city.
They show that the resulting cities generated tend to be closer in distance to the reference point when the ``near'' prompt is used, and farther when the ``far'' prompt is used, indicating LLMs are capable of basic spatial reasoning in that they associate metric (distance) meaning to those keywords.



\subsection{Directional Relations}
It is highly likely that popular cities are written about repeatedly in the vast training data LLMs are exposed to.
As a result, we would expect LLMs to memorize many facts about these cities, including potentially where one is located with respect to another, if that was explicitly stated in the data.
However, it is much less likely that the relative locations of three cities with respect to each other appears explicitly in the training data.
Hence, we believe the poor performance across most of the three-way directional relations we tested indicates a lack of the model's ability to infer implicit spatial relationships that it did not learn directly from training.
We discuss ways complex spatial relations can be encoded and provided to models at training time in section \ref{section:future}.


\subsection{Topological Relations}



\subsection{Order Relations}
