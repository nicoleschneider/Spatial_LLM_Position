
Spatial reasoning is a particularly challenging form of reasoning that requires inferring implicit information about objects based on their relative positions in space.
Traditionally, spatial reasoning is addressed using formal methods that rely on pre-computed indices and data structures, which limit the scope of questions that can be answered.
As the research community moves towards developing general purpose geo-foundation models that can perform a variety of spatial reasoning tasks, recent research has explored what kinds of world knowledge and spatial reasoning capabilities Large Language Models (LLMs) naturally inherit from their training data.
In this paper we assess the spatial reasoning ability of LLMs through a set of experiments designed to cover a broad range of spatial tasks, including toponym resolution, and reasoning about four fundamental spatial relations: metric, directional, topological, and order relationships.
While previous work has demonstrated that LLMs have some basic level of spatial awareness in the form of knowledge about geocoordinates, directional relationships between major cities, and distances between cities, we find that increasing the complexity of spatial tasks to include more than two entities and expanding the coverage to address all four major forms of spatial relations reveals significant gaps in the spatial reasoning abilities of the LLMs tested.
Given these findings, we propose several avenues of opportunity to improve the spatial reasoning ability of LLMs.



%LLMs are used for increasingly complex tasks, including ones grounded in the physical world, like generating routes between known points or suggesting places of interest to a user based on their location and trajectory.

% We describe several embedding techniques and ideas for self-supervised training objectives that may address these shortcomings and enable LLMs to more effectively learn to reason over spatial data.
% Finally, we discuss longer term opportunities with multimodal learning that may ultimately enable the design of a general geo foundation model that can perform spatial reasoning over a variety of sources and scales of geospatial data.
