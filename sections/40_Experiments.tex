\section{Experiments}
\label{section:experiments}

To illustrate the challenges associated with training an LLM to perform spatial reasoning, we design a set of experiments to probe Chat-GPT \nrscomment{cite} for its spatial reasoning abilities.
We first probe its reasoning ability across targeted spatial entity types, and then its ability to handle various complex spatial relations.


\subsection{Entities} % -----------------------------------------------

\subsubsection{Non-point Data}
\cite{Liu2023} can do NL2Spatial Query which can handle region/line data, but can an LLM as a db handle it?
--> Show this info isn't learned in LLMs by repeating queries similar to \cite{Liu2023} nanjingtest and berlintest region and line queries but instead give them to ChatGPT.

\subsubsection{Lesser-known Cities}
--> Show correlation between cities recognized and their population size.



\subsection{Relations} % -----------------------------------------------

\subsubsection{N-way Spatial Relations}
--> Repeat AUS cities test from \cite{Qi2023} paper but with 3 way relations to show it isn't really doing spatial reasoning

\subsubsection{Non-directional Spatial Relations}
Cyclic? Topological?
--> Show this info isn't learned in LLMs by making queries similar to the \cite{Qi2023} ones but for relations types besides directional.

\subsubsection{Spatiotemporal and Multiple Hop Relations}
--> Test if Chat-GPT can answer queries like
- whether event x happened north of event y (2 hops event->loc + loc->spatial)
- all events in x region that happened between y and z dates (intersect space and time)
- all events in x region that happened between y and z dates north of location A (spatiotemporal involving spatial relation)








        
% Future Work -------------------------------------

% \subsection{SPM}
% \nrscomment{Experiment.}
% --> Try giving Chat-GPT a bunch of points A, B, C with coords like the pictorial query grid has and then asking it which cities in a given region match that pattern.
% Also try giving the input instead as a list of pairwise constraints.
% Use OSM to figure out the ground truth by pulling all city tags in the same region and running a traditional SPM algorithm to find all matching patterns.
% Report precision and recall for both input types.

% Check - can it produce an image from the points given? Can we give it an image with points as input?



% \paragraph{Locations being too close to differentiate in the embedding space}
% <ref from translation clustering paper>
% \nrscomment{Experiment.}
% --> Design tests to demonstrate these issues - similar to above, compare the embeddings of locations.