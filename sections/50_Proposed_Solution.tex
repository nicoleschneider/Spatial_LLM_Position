\section{Proposed Solution}
\label{section:proposal}

In this section we discuss ways to address the spatial reasoning limitations of LLMs that were uncovered by our experiments in the previous section.


\subsection{Embedding Techniques}
We first envision the development of novel embedding methods and self-supervised training objectives for various types of geodata.
%
% \subsubsection{Spatial Heterogeneity} % R1 non-point data usually visual
% For RSI data, standard vision embeddings and convolution architectures can be leveraged to encode the visual data and handle autocorrelation naturally present in the data~\cite{Xie2021}.
% However, spatial heterogeneity remains a challenge most neural pipelines are unequipped to handle.
% To address this, the model design or embedding schema must be adapted to account for the varying spatial processes driving the heterogeneity of the underlying data.
% \nrscomment{what are we suggesting beyond what Spatial-Net already did?}
%
%\subsubsection{Complex spatial relations} % R3-4 n-way and beyond directional rels, R5 multihop with paths
To allow LLMs to learn complex spatial relations, such as multi-way directional relations that we showed to be a shortcoming in Experiment 1, we propose using an appropriate encoding scheme for that type of information.
In the spatial pattern matching domain, complex spatial relationships are captured using graph encodings, where relations can be made explicit using the edges between graph nodes.
For complex spatial data, a fully connected graph or multigraph may be needed to capture all the relevant relations. 
Once encoded, spatial graph data can be embedded using node-level, edge-level, community-level, and graph-level embeddings that are typically used in graph learning approaches like Graph Neural Networks (GNNs)~\cite{Bai2019,Krlevza2016,Liu2020Neural}.
By including both local context embeddings at the node and community level and global context embeddings at the graph level, the LLM would have adequate context for learning the spatial relationships captured in the data.


% \subsubsection{Structured geodata}  % R5 multihop via table joins or temporal via table info
% For structured tabular geodata, the embedding techniques like row and column embeddings that are commonly used with non-spatial tabular transformers offer an initial way forward.
% However, additional work will be needed to adequately handle geo entities that would otherwise be treated as out-of-dictionary tokens within table cells.
% \nrscomment{suggest a training objective that would fit well with geoentity tables, look at NALSpatial}

% \nrscomment{A note about generalizability and dealing with less popular cities that are OOV or not seen at training time} % R2


\subsection{Self-Supervised Training Tasks}
Once embedding methods have been well-established, self-supervised training objectives are needed to learn the embeddings during pre-training.
%
Some initial methods have been proposed for pre-training spatial coordinate embeddings that encode latitude and longitude coordinates for textual references to geographic entities~\cite{Li2021}.
In their paper, \citeauthor{Li2021} craft pseudosentences that list geoentities in descending order of distance to an entity of interest, and then employ self-supervised masked entity prediction and masked subtoken prediction objectives for pre-training.
Although this method shows improvement over non-spatial methods at the downstream tasks of entity type classification and entity linking, there is room for improvement by developing more intuitive pre-training tasks that more thoroughly capture two-dimensional spatial relationships (as opposed to a one-dimensional distance metric).

For example, we can leverage natural language descriptions of the spatial relationships between neighboring entities including words like ``northeast of'', ``adjacent to'', ``left of'' that might be applicable in downstream spatial reasoning tasks.
Critically, this approach may help to address the issue we observed in experiment 2, where the LLM lacked knowledge about the relative positions of cities.

Additionally, we envision that these natural language descriptions can be augmented through the application of spatial logic to generate additional inputs that help ensure the model's spatial reasoning is self-consistent (an issue we observed in Experiment 3).
For instance, the phrase ``A is northeast of B'' implies that ``B is southwest of A'' is also true.
By exposing the model to both forms of the spatial relationship, it may more readily learn the logical link between ``northeast'' and ``southwest''.
This technique can be applied to a variety of spatial relationships, including cyclic order relations and topological relations.
It could also be adapted to train the graph embeddings proposed above, where relations like ``northeast'' would be encoded as a link between A and B, and where spatial logic dictates, another link in the reverse direction could be added.

\nrscomment{come up with an actual idea for the graph self-supervised objective that stands on its own}





\subsection{Long Term Opportunities: Multimodal Spatial Learning}
Once LLMs can be successfully pre-trained on these various types of spatial data with success in downstream tasks, it will then be possible to leverage work in multimodal learning to combine a variety of input modalities of geospatial data.
Doing so would enable the training of a more generic geo foundation model broadly capable of spatial reasoning given new sources of spatial information, at a variety of scales.
To accomplish this goal, multimodal learning techniques that have been successful at enabling question answering over visual, textual, and other types of data may be worth adapting to the spatial domain~\cite{Fei2022}.

\nrscomment{be more explicit with details from multimodal learning lit}