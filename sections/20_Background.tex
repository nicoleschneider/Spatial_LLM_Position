\section{Spatial Reasoning}
\label{section:background}

In this section we define the problem of spatial reasoning and describe how it presents a worthwhile challenge to address using LLMs.

\subsection{Preliminaries}
\nrscomment{Define spatial reasoning over spatial entities and spatial relations.}

\nrscomment{Explain how spatial reasoning is done traditionally?}


\subsubsection{Spatial Entities}
Spatial entities are the basic elements of spatial data.
They can be classed into three main types: \textbf{(i)} \textit{Points}, which consist of an $(x,y)$ coordinate pair in Cartesian space, \textbf{(ii)} \textit{Lines}, which represent the shortest path between two points, and \textbf{(iii)} \textit{Regions}, which represent the area inside a polyline joining several points. 
Point entities are typically used to represent locations in the world, like landmarks, towns, buildings, and objects.
Line entities are typically used to represent ways, like roadways, waterways, and railways.
Region entities are typically used to represent the extent of lakes, stadiums, large buildings, states, and counties.


\subsection{Spatial Relations}
\textit{Points, Lines} and \textit{Regions} relate to each other spatially through the following types of relationships~\cite{Carniel2020,Bertella2022,Carniel2023}: 
\textbf{(i)} \textit{Metric relations} that describe the distances between spatial entities (like ten miles, near, far), 
\textbf{(ii)} \textit{Topological relations} that describe how regions, lines, and points interact (like intersect, contain, touch), and 
\textbf{(iii)} \textit{Directional relations} that describe how entities are positioned relatively in space (like north, left, behind),
\textbf{(iv)} \textit{Order} relations that describe the cyclic order in which objects appear with respect to a central coordinate~\cite{Schwering2014}.
%
Hybrid relations that combine two or more of the primitive spatial relations are also possible~\cite{Carniel2023}.



\subsection{Motivation}
Why is this an important problem that machine learning people should care about solving?

What's the motivating use case?


\subsection{Challenges}
What makes spatial reasoning easy for humans but difficult for machines?


\subsection{LLM Spatial Reasoning}
How have people tried to use LLMs to aid/do it?

What are the key challenges that haven't been addressed?