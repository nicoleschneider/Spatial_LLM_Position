\section{Spatial Reasoning}
\label{section:background}

In this section we define the problem of spatial reasoning and describe how it presents a worthwhile challenge to address using LLMs.

\subsection{Preliminaries}
Spatial reasoning is the task of reasoning about entities or objects that are located in space and possess spatial structure, or have spatial relationships with each other~\cite{Varzi2007}.
The abstract theories that define the core principles of spatial reasoning are commonly divided into classes including 
\textbf{(\textit{i})} \textit{mereology}, the theory of parthood relations;
\textbf{(\textit{ii})} \textit{topology}, the theory of qualitative spatial relations such as continuity and contiguity, and 
\textbf{(\textit{iii})} the \textit{theory of location proper}, which deals with the relationship between an entity and the spatial region it
occupies~\cite{Varzi2007}.

In this paper we discuss spatial reasoning tasks, which involve answering questions about certain relationships (termed spatial relations) between some set of objects in space.



\subsubsection{Spatial Entities}
Spatial entities are the basic elements of spatial data.
They can be classed into three main types: \textbf{(\textit{i})} \textit{Points}, which consist of an $(x,y)$ coordinate pair in Cartesian space, \textbf{(\textit{ii})} \textit{Lines}, which represent the shortest path between two points, and \textbf{(\textit{iii})} \textit{Regions}, which represent the area inside a polyline joining several points. 
Point entities are typically used to represent locations in the world, like landmarks, towns, buildings, and objects.
Line entities are typically used to represent ways, like roadways, waterways, and railways.
Region entities are typically used to represent the extent of lakes, stadiums, large buildings, states, and counties.


\subsection{Spatial Relations}
\textit{Points, Lines} and \textit{Regions} relate to each other spatially through the following types of relationships~\cite{Carniel2020,Bertella2022,Carniel2023}: 
\textbf{(\textit{i})} \textit{Metric relations} that describe the distances between spatial entities (like ten miles, near, far), 
\textbf{(\textit{ii})} \textit{Topological relations} that describe how regions, lines, and points interact (like intersect, contain, touch), and 
\textbf{(\textit{iii})} \textit{Directional relations} that describe how entities are positioned relatively in space (like north, left, behind), and
\textbf{(\textit{iv})} \textit{Order} relations that describe the cyclic order in which objects appear with respect to a central coordinate~\cite{Schwering2014}.
%
Hybrid relations that combine two or more of the primitive spatial relations are also possible~\cite{Carniel2023}.



\subsection{Motivation and Challenges}
Humans seamlessly apply spatial reasoning to the objects we perceive in our environment every day.
Without consciously realizing it, we may use spatial reasoning to navigate from one place to another, avoid colliding with other moving entities, or recall and describe a place by the spatial configuration of its nearby landmarks.
To do this we rely on world knowledge and experience, spatial intuition, common sense, and embodiment.
Spatial reasoning is a particularly challenging form of reasoning since it is often a dynamic process that involves gathering data that may be changing in real time as objects move through space.
Likewise, the perspective from which the data is gathered may change -- either as the collection source moves, or as data is combined from various sources, at various scales (i.e. overhead imagery, motion detection sensors, dashboard cameras).
However, even with advances in machine perception and massive models with enough parameters to memorize world knowledge, spatial reasoning remains a challenge that machine learning has yet to solve.

Traditionally, many spatial reasoning tasks like spatial pattern matching (determining which objects in the world match a set of query constraints) can be addressed using formal methods, such as by formulating qualitative constraint networks and solving them using constraint satisfaction approaches~\cite{Papadias1998, Schwering2014, Duckham2023}.
Isomorphic subgraph matching approaches are also common~\cite{Folkers2000, Chen2019, Fang2019}.
However, these traditional methods are rigid and computationally slow, often relying on pre-computed indices and data structures, and tend to be limited to handling only point entities, or directional constraints.



\subsection{Machine Learning for Spatial Reasoning}
Although some work has been done using machine learning to approximately solve generic (non-spatial) subgraph matching problems~\cite{Krlevza2016, Liu2020Neural,Lan2021,Roy2022}, the approaches are limited to small or undirected graphs and are not yet robust enough to use for spatial reasoning tasks at scale.
One of the reasons spatial reasoning is computationally difficult is that even on a small scale, the density of information required to capture all the explicit spatial relations between entities is high.
This makes search and reasoning over spatial data challenging, necessitating flexible, approximate approaches.
The need for flexibility makes machine learning a good candidate to solve spatial reasoning problems.
Along the same lines, large models, like LLMs, must perform well at a variety of tasks that are grounded in the physical world.
Whether it is generating a route between known points or suggesting places of interest to a user based on their location and trajectory, general purpose machine learning models need to perform some degree of spatial reasoning.
However, it has previously been unclear to what degree general purpose foundation models could reason about implicit spatial relationships that were not explicitly found in their training data.
We address this through a set of experiments that probe the spatial reasoning capabilities of an LLM.

%\osullikomment{There's a personal assistant angle to this "Hey Siri, what's the name of the place off the interstate with all those dinosaur statues out back", but that's a little general. For ML Theory you might be able to make an explainability / interpretability argument. Hmm, I feel like there's more though. Ask me again later}



% \subsection{LLM Spatial Reasoning}
% How have people tried to use LLMs to aid/do it?

% What are the key challenges that haven't been addressed?
